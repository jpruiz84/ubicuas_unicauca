\documentclass[10pt,letterpaper]{article}

\usepackage{amsmath}

\usepackage[T1]{fontenc} 
\usepackage[utf8x]{inputenc}
\usepackage[spanish]{babel}
\usepackage{graphicx}% Include figure files

%%% Margins
\oddsidemargin=-0.54cm
\textwidth=17.6cm
\topmargin=-2.54cm
\headheight=0cm
\headsep=2cm
\textheight=23.8cm

\makeatletter
\renewcommand\section{\@startsection{section}{1}{-15pt}{20pt}{10pt}{\textsc}}
\makeatother

\makeatletter
\renewcommand\subsection{\@startsection{subsection}{2}{0pt}{0pt}{0pt}{\textit}}
\makeatother


\begin{document}

\begin{tabular}{p{1.3in}p{6in}}
\begin{flushleft}
\noindent \includegraphics[bb = 2.5cm 0cm 10.29cm 9.78cm,scale=0.2]{escudoUnicacuaSolo.eps}
\end{flushleft} &
\normalsize \vspace{0.6cm}
\textsc{Universidad del Cauca}

\textsc{Facultad de Ingeniería Electrónica y Telecomunicaciones}

%\textsc{Maestría en Ingeniería Telemática}
\textsc{Programa de Ingeniería Electrónica y Telecomunicaciones}

\textsc{Desarrollo de aplicaciones para plataformas ubicuas}

\end{tabular}
\begin{tabbing}
\hspace{3cm} \= \hspace{5.3cm} \= \hspace{6cm} \kill
\textbf{Docente}: \> Juan Pablo Ruiz Rosero			\> \texttt{jpabloruiz@unicauca.edu.co} \\
\end{tabbing}

\begin{center}
%\textbf{1. Lenguaje de programación de alto nivel Python}
\textbf{1. Introducción a Python}
\end{center}

\begin{enumerate}
\item Abra la consola o linea de comandos de su sistema operativo (Widnows: Ctrl+R, cmd, Enter)
\item Entre a la consola python escribiendo en la linea de comandos:\\
\verb+python+
\item Entre a la consola python escribiendo en la linea de comandos:\\

\item Mediante la función print, pruebe la impresión de mensajes de depuración así:\\
\verb+print("Hola mundo")+

\item Defina una variable con su nombre, para python lo salude:
\begin{verbatim}
name = "Juan"
print("Hola " + name)
\end{verbatim}
\begin{itemize}
\item Las variables en python no se inicializan, solo se definen.
\item La concatenación de string se hace con el +
\end{itemize}

\item Escriba un script que salude a un usuario con la fecha y hora:
\begin{verbatim}
from datetime import datetime
print(raw_input("Digite su nombre: ") + " ha entrado a las: " + str(datetime.now()))
\end{verbatim}
\begin{itemize}
\item La inclusión de librerías se realiza mediante import, from especifica que sub librería se requiere importar.
\item La función \verb+raw_input()+ entrega la linea de texto introducida por el usuario.
\item La función \verb+datetime.now()+ entrega la fecha y hora en formato datetime, str la convierte a texto.
\end{itemize}

\item Salga de la consola python con la función quit():\\
\verb|quit()|


\item Cree un archivo llamado guia01a.py, que le informe si usted es mayor o menor de edad:
\begin{verbatim}
edad = raw_input("Introduzca su edad: ")
if int(edad) >= 18:
  print("Usted es mayor de edad")
else:
  print("Usted es menor de edad")
\end{verbatim}
\begin{itemize}
\item Los delimitadores de funciones o secuencias logicas en python son los : y las sangrías (tabulaciones o espacios). 
\item La función \verb|int()| permite convertir de string a numero entero
\item Ejecute este script de la siguiente manera: \\
\verb|python guia01a.py|
\end{itemize}

\item Cree un archivo llamado guia01b.py, que le calcule su masa corporal.

\item Cree un archivo llamado guia01c.py para llenar una lista de usuarios.
\begin{verbatim}
userList = []

while True:
  user = raw_input("Introduzca el usuario, 'fin' para terminar:")
  if user == "fin":
    break
  userList.append(user)

count = 0
for user in userList:
  print(str(count) + ". " + user)
  count += 1
\end{verbatim}

\item Agregue el modificador en guia01c.py \verb|.upper()| a los strings, para aceptar el string fin,Fin, o FIN para terminar.

\item En la consola de python, imprima el numero de segundos desde el 1 de Enero de 1970 (Unix time), imprima ese valor en formato de hora local, agregue 1 millon de segundos a ese valor y vuelva a imprimirlo en formato de hora local.
\begin{verbatim}
import time
time1 = time.time()
print(str(time1))
print time.strftime("%Z - %Y/%m/%d, %H:%M:%S", time.localtime(time1))
\end{verbatim}

\item Modifique el programa guia01c.py de tal forma, que se genere una lista \verb|userInputTime| con la fecha en que se introdujo ese usuario. Imprima al final el nombre de cada usuario con su respectiva fecha de ingreso.

\item Modifique el programa guia01c.py de tal forma, que si el usuario introduce un nombre 20 o más segundos más tarde que el último, la lista termine.


\end{enumerate}


\end{document} 