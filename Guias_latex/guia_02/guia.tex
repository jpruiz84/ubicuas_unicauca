\documentclass[10pt,letterpaper]{article}

\usepackage{amsmath}

\usepackage[T1]{fontenc} 
\usepackage[utf8x]{inputenc}
\usepackage[spanish]{babel}
\usepackage{graphicx}% Include figure files

%%% Margins
\oddsidemargin=-0.54cm
\textwidth=17.6cm
\topmargin=-2.54cm
\headheight=0cm
\headsep=2cm
\textheight=23.8cm

\usepackage{fancyvrb,xcolor}


\makeatletter
\renewcommand\section{\@startsection{section}{1}{-15pt}{20pt}{10pt}{\textsc}}
\makeatother

\makeatletter
\renewcommand\subsection{\@startsection{subsection}{2}{0pt}{0pt}{0pt}{\textit}}
\makeatother

\makeatletter 
\renewcommand\verbatim@font{\normalfont\ttfamily\color{blue}}
\makeatother

\begin{document}

\begin{tabular}{p{1.3in}p{6in}}
\begin{flushleft}
\noindent \includegraphics[bb = 2.5cm 0cm 10.29cm 9.78cm,scale=0.2]{escudoUnicacuaSolo.eps}
\end{flushleft} &
\normalsize \vspace{0.6cm}
\textsc{Universidad del Cauca}

\textsc{Facultad de Ingeniería Electrónica y Telecomunicaciones}

%\textsc{Maestría en Ingeniería Telemática}
\textsc{Programa de Ingeniería Electrónica y Telecomunicaciones}

\textsc{Desarrollo de aplicaciones para plataformas ubicuas}

\end{tabular}
\begin{tabbing}
\hspace{3cm} \= \hspace{5.3cm} \= \hspace{6cm} \kill
\textbf{Docente}: \> Juan Pablo Ruiz Rosero			\> \texttt{jpabloruiz@unicauca.edu.co} \\
\end{tabbing}

\begin{center}
%\textbf{2. Estructuras de datos en alto nivel.}
\textbf{2. Diccionarios en Python}
\end{center}

Python ofrece un tipo de dato llamado diccionario para almacenar de manera sencilla y eficiente estructuras de datos en alto nivel. Un Diccionario es una estructura de datos y un tipo de dato en Python con características especiales que nos permite almacenar cualquier tipo de valor como enteros, cadenas, listas e incluso otras funciones. Estos diccionarios nos permiten además identificar cada elemento por una clave (Key).

\begin{enumerate}
\item En la consola python defina un diccionario de la siguiente manera: 
\begin{verbatim}
d1 = {"nombre" : "Pedro", "edad" : 22, "cursos": ["Python","C++","Java"] }
\end{verbatim}

\item Imprima el nombre y los cursos almacenados en el diccionario d1
\begin{verbatim}
print(d1["nombre"])
print(d2["cursos"])
\end{verbatim}

\item Cree una lista de diccionarios que contenga el nombre, la edad y los cursos que han tomado cuatro de sus compañeros, guarde este script en el archivo guia02a.py, ejemplo: 
\begin{verbatim}
studentsList = []

studentIn = {"nombre" : "Juan", "edad" : 19, "cursos": ["Basic"] }
studentsList.append(studentIn)
studentIn = {"nombre" : "Marcos", "edad" : 20, "cursos": ["Turbo C","Pascal"] }
studentsList.append(studentIn)
\end{verbatim}


\item En el archivo guia02a.py, añada al final un ciclo for que imprima el nombre, la edad y los cursos de cada uno de los objetos de la lista studentsList.

\item En el archivo guia02a.py, añada la librería csv (\verb|import csv|) para exportar los datos a un archivo de texto separado por comas (csv). Agregue al final del guia02a.py las siguientes lineas para generar el archivo de salida csv
\begin{verbatim}
with open("outputFile.csv", "wb") as f:
  w = csv.DictWriter(f, studentsList[0].keys())
  w.writeheader()
  for student in studentsList:
    w.writerow(student))
\end{verbatim}

\item Con base al archivo test3.py de la guía anterior, cree un nuevo archivo guia02b.py, en el cual el script pida los siguientes datos a cada usuario para guardarlos en una lista de diccionarios: 
\begin{itemize}
\item Nombres
\item Apellidos
\item Edad
\end{itemize}

Añada al diccionario el campo tiempoDeRegistro, donde se añada autómaticamente la fecha y la hora del registro del usuario. Al ingresar el usuario con el nombre Fin, termine el registro de usuarios y guarde la lista de diccionarios en el archivo usuarios.csv


\item Cree un archivo nuevo guia02c.py, el cual abra los datos del archivo outputFile.csv en el diccionario studentsList y los imprima en la consola mediante un ciclo for.
\begin{verbatim}
import csv

input_file = csv.DictReader(open("outputFile.csv"))

studentsList = []
for row in input_file:
  studentsList.append(row)

# Agruegue el for loop para imprimir datos  
\end{verbatim}

\item Con base al archivo guia02b.py cree el archivo guia02d.py, el cual registre usuarios, los guarde en el archivo outputFile.csv pero conserve los registros almacenados anteriormente. 

\item Modifique el archivo guia02d.py de tal modo que no permita introducir usuarios duplicados (que contengan el mismo nombre y apellido).






\end{enumerate}




\end{document} 