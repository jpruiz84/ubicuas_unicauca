\documentclass[10pt,letterpaper]{article}

\usepackage{amsmath}

\usepackage[T1]{fontenc} 
\usepackage[utf8x]{inputenc}
\usepackage[spanish]{babel}
\usepackage{graphicx}% Include figure files

\usepackage{fancyvrb,xcolor}
\usepackage{upquote,textcomp}

%%% Margins
\oddsidemargin=-0.54cm
\textwidth=17.6cm
\topmargin=-2.54cm
\headheight=0cm
\headsep=2cm
\textheight=23.8cm

\makeatletter
\renewcommand\section{\@startsection{section}{1}{-15pt}{20pt}{10pt}{\textsc}}
\makeatother

\makeatletter
\renewcommand\subsection{\@startsection{subsection}{2}{0pt}{0pt}{0pt}{\textit}}
\makeatother

\usepackage{hyperref} % for urls


% Default fixed font does not support bold face
\DeclareFixedFont{\ttb}{T1}{txtt}{bx}{n}{8} % for bold
\DeclareFixedFont{\ttm}{T1}{txtt}{m}{n}{8}  % for normal

% Custom colors
\usepackage{color}
\definecolor{deepblue}{rgb}{0,0,0.5}
\definecolor{deepred}{rgb}{0.6,0,0}
\definecolor{deepgreen}{rgb}{0,0.5,0}
\definecolor{deepgray}{rgb}{0.5,0.5,0.5}

\usepackage{listings}

\lstdefinestyle{myCustomPythonStyle}{
	language=Python,
	basicstyle=\linespread{1}\small,
	otherkeywords={self},             % Add keywords here
	keywordstyle=\ttb\color{deepblue},
	emph={MyClass,__init__},          % Custom highlighting
	emphstyle=\ttm\color{deepred},    % Custom highlighting style
	commentstyle=\ttm\color{deepgray},
	stringstyle=\color{deepgreen},
	frame=tb,                         % Any extra options here
	showstringspaces=false,
	columns=fullflexible,
	keepspaces=true,
	literate={-}{-}1,				% To avoid problems with hyphen -
}



\makeatletter 
  \renewcommand\verbatim@font{\normalfont\ttfamily\color{blue}}
\makeatother


\begin{document}
% To fix ' no curve quotation mark
\newcommand\upquote[1]{\textquotesingle#1\textquotesingle}



\begin{tabular}{p{1.3in}p{6in}}
\begin{flushleft}
\noindent \includegraphics[bb = 2.5cm 0cm 10.29cm 9.78cm,scale=0.2]{escudoUnicacuaSolo.eps}
\end{flushleft} &
\normalsize \vspace{0.6cm}
\textsc{Universidad del Cauca}

\textsc{Facultad de Ingeniería Electrónica y Telecomunicaciones}

%\textsc{Maestría en Ingeniería Telemática}
\textsc{Programa de Ingeniería Electrónica y Telecomunicaciones}

\textsc{Desarrollo de aplicaciones para plataformas ubicuas}

\end{tabular}
\begin{tabbing}
\hspace{3cm} \= \hspace{5.3cm} \= \hspace{6cm} \kill
\textbf{Docente}: \> Juan Pablo Ruiz Rosero			\> \texttt{jpabloruiz@unicauca.edu.co} \\
\end{tabbing}

\begin{center}
\textbf{8. Gestion de buses por beacons Bluetooth}
\end{center}
\section{Objetivo}
Esta guía tiene como objetivo el desarrollo de una aplicación web y un servicio capaz de trazar la ruta de un bus por diferentes estaciones. 

\section{Descripción}
Suponga que quiere trazar la ruta de un bus, el cual cuenta con un computador portátil capaz de escanear beacons Bluetooth. El bus debe pasar por 4 estaciones equidistantes, separadas 1Km cada una, donde cada estación tiene un beacon Bluetooth. 

\section{Entradas}
\begin{itemize}
\item Un archivo csv denominado \verb|beaconsStations.csv|, el cual contiene la dirección MAC de los beacons relacionada con el nombre de la ruta, y el orden de paso.
\begin{center}
\begin{tabular}{|l|l|c|}
\hline
\textbf{BeaconMAC} & \textbf{Station} & \textbf{Order} \\ \hline
C5:F8:14:9F:25:D2  & Piedra sur  & 1   \\ \hline
DD:EE:CC:74:3B:DA  & INEM    & 2  \\ \hline
D4:5B:96:45:BA:02  & Carantanta  & 3   \\ \hline
DC:82:79:2A:C1:AB  & Campanario  & 4  \\ \hline
\end{tabular}

\end{center}
\item Servicio de escaneo de beacons BLE en intervalos de 5 segundos, el cual registre solo el beacon con mayor RSSI, y RSSI mayor a -60dBm.
\end{itemize}

\section{Salidas}
\begin{itemize}
\item Una base de datos denominada \verb|stops.db3| en donde se registre la fecha y hora de llegada del bus a cada estación.\\

\item Una aplicación web con los siguientes servicios:
\begin{itemize}
\item Consulta de la fecha y hora de todas las paradas mediante la dirección web \url{http://127.0.0.1:5000}
\item Consulta de todas las paradas de un día determinado mediante la url\\ \url{http://127.0.0.1:5000/date/2017_09_13}\\
Nota: formato de fecha año\_mes\_día
\item Consulta del la última parada mediante la url\\ \url{http://127.0.0.1:5000/lastStop}\\
\item Consulta del la velocidad actual estimada del bus\\ \url{http://127.0.0.1:5000/speed}\\
\item Consulta de la siguiente parada y minutos estimados para la llegada mediante la url  \\ \url{http://127.0.0.1:5000/nextStop}\\
\end{itemize}

\end{itemize}





\end{document} 